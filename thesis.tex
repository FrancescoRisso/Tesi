% !TEX encoding = UTF-8 Unicode
% !TEX TS-program = pdflatex

% toptesti document class
\documentclass[%
    a4paper, % not needed, by default it is a4paper, or also b5paper can be used
    corpo=10pt, % dimension of basic font
    % oneside is generally the way to go
    oneside, % two side optimizes for two-face printing, having chapters open on the right (aka odd numbers), if you don't want blank pages put oneside here
    stile=standard,
    %evenboxes, % not needed, to put supervisors and candidate at the same level
    tipotesi=magistrale,
    numerazioneromana, % roman numbering for appendixes and preambles, up to Table of Contents
    openright, % to force opening on the right for double-sided printing
    cucitura=7mm, % for printing, 7mm should be enough
    %dvipsnames, % for compatibility with xcolor, it does not work
]{toptesi}

%%%%%%%%%%%%%%%%%%%%%%%%%%%%%%%%%%%%%%%%%%%%%%%%%%%%
\usepackage[english]{babel}
\usepackage[utf8]{inputenc}
\usepackage[T1]{fontenc}
\usepackage{lmodern}

\usepackage{hyperref} % must be loaded before glossaries-extra

% bibliography
\usepackage[hyperref=true,backref=true,backend=biber,maxbibnames=9,maxcitenames=2,style=numeric,citestyle=numeric,sorting=none]{biblatex} % hyperref uses links, backref goes back to citations, uses biber as backend, with 9 names at most in bibliography and 2 in citations, citing using numbers, and sorting in citation order
% sorting can be also ydnt for year descending, name, title or ynt for ascending year

\usepackage{adjustbox} % to resize boxes by keeping the same aspect ratio
\usepackage{algorithm} % algorithm environment
\usepackage{algpseudocode} % improved pseudo-code
\usepackage{amsfonts}               %  AMS mathematical fonts
\usepackage{amsmath}
\usepackage{amssymb}                %  AMS mathematical symbols
\usepackage{bm}                     %  black/bold mathematical symbols
\usepackage{booktabs}               %  better tables
\usepackage[labelfont=bf]{caption} % font=footnotesize % to have reduced caption font size
\usepackage{csquotes}
\usepackage{enumitem} %left align the bulleted points
\usepackage{geometry}
%\usepackage{glossaries} % to use acronyms and glossary, it has also glossaries-extra as extension, but commands are different
\usepackage[%
    toc, % puts the link in the ToC
    %record, % to use bib2gls
    abbreviations, % to load abbreviations / acronyms
    nonumberlist, % to avoid printing the numbers of the references in the acronyms page
]{glossaries-extra}
\usepackage{graphicx}               %  post-script images
%\usepackage{iwona} % extra fonts, substitute standard ones
\usepackage{listings} % to insert formatted code
\usepackage{lipsum} % for lorem ipsum text, not needed in the real work
\usepackage{makecell} % to change dimensions of cells, for math cases
\usepackage{mathtools} % for additional commands
\usepackage{mfirstuc} % to have capitalization capabilities
\usepackage[final]{microtype}      % microtypography, final lets latex use it also in bibliography
\usepackage{multirow} % to allow for cells covering more than 1 row in tables
\usepackage{nicefrac}       % compact symbols for 1/2, etc.
%\usepackage[lofdepth,lotdepth]{subfig}
\usepackage{ragged2e} % for justifying text
\usepackage{siunitx} % support for SI units of measurement and number typesetting
\usepackage{subfig}
\usepackage{svg} % for svg support, works only if inkscape is installed, default for Overleaf v2
%\usepackage{subfigure}              %  subfigure compatibility, can be removed if subfig
\usepackage{tabularx} % equal-width columns in tables
\usepackage{textcomp} % extra fonts and symbols
\usepackage{url}            % simple URL typesetting
\usepackage{verbatim} % for extended verbatim support
\usepackage{xcolor} % to define colors and use standard CSS names add dvipsnames as option, but it clashes with xcolor loaded in toptesi, pay attention that if it goes in conflict with tikz/beamer, simply use \documentclass[usenames,dvipsnames]{beamer}, along with other custom options when defining the document class


% configuration for glossaries
% convert and load converted glossaries in .tex ,format from .bib
\setabbreviationstyle{long-short-desc} % style before loading resources
% this command sets the style to title for long names of acronyms only in the glossary description, leading to capitalized first-letter for all words
% \glssetcategoryattribute{\glsxtrabbrvtype}{glossname}{capitalisewords} % doesn't work
% resources to load if using a bib file with bib2gls
%\GlsXtrLoadResources[%
% src={glossaries}, % name of the file without extension
% selection=all, % select all the entries
%]
% not needed
%\newglossary*{abbreviation}{Acronyms} % to change the name of this glossary for acronyms

%\renewcommand{\glsclearpage}{\paginavuota} % to allow glossaries to clear pages, done manually is better


% setup for hyperref
\hypersetup{%
    pdfpagemode={UseOutlines},
    bookmarksopen,
    pdfstartview={FitH},
    colorlinks,
    linkcolor={black}, % it is suggested to keep them black, since when printing it it costs per page, and if they have color it's twice the price per page
    citecolor={black},
    urlcolor={black}
  }
%

% setup for svg
\svgsetup{%
    inkscapeformat=pdf, % to force usage of PDF
    inkscapelatex=false, % to disable latex rendering of text, produces errors
}

% setup for siunitx, it does not work in the summary
\sisetup{%
    detect-all, % to use the same font as for writing when using \num
    mode=text, % to allow it to work also in math mode
    group-separator = {,}, % separator for number grouping
    group-minimum-digits = 3, % minimum number of digits a number must have to be grouped in 3-digit groups
}

% listings colours
\definecolor{rulecolor}{rgb}{0,0,0}
\definecolor{commentcolor}{rgb}{0,0.6,0}
\definecolor{linenumbercolor}{rgb}{0.5,0.5,0.5}
\definecolor{keywordcolor}{rgb}{0,0,0.95}
\definecolor{backcolor}{rgb}{1,1,1}%{0.95,0.95,0.92}
\definecolor{stringcolor}{rgb}{0.58,0,0.82}

% setup for lstlisting
\lstset{ %
	backgroundcolor=\color{backcolor},   % choose the background color; you must add \usepackage{color} or \usepackage{xcolor}; should come as last argument
	basicstyle=\footnotesize,        % the size of the fonts that are used for the code
	breakatwhitespace=false,         % sets if automatic breaks should only happen at whitespace
	breaklines=true,                 % sets automatic line breaking
	captionpos=t,                    % sets the caption-position to bottom
	commentstyle=\color{commentcolor},    % comment style
	extendedchars=true,              % lets you use non-ASCII characters; for 8-bits encodings only, does not work with UTF-8
	frame=single,	                   % adds a frame around the code
	keepspaces=true,                 % keeps spaces in text, useful for keeping indentation of code (possibly needs columns=flexible)
	keywordstyle=\color{keywordcolor},       % keyword style
	%language=VHDL,                 % the language of the code
	numbers=left,                    % where to put the line-numbers; possible values are (none, left, right)
	numbersep=5pt,                   % how far the line-numbers are from the code
	numberstyle=\tiny\color{linenumbercolor}, % the style that is used for the line-numbers
	rulecolor=\color{rulecolor},         % if not set, the frame-color may be changed on line-breaks within not-black text (e.g. comments (green here))
	showspaces=false,                % show spaces everywhere adding particular underscores; it overrides 'showstringspaces'
	showstringspaces=false,          % underline spaces within strings only
	showtabs=false,                  % show tabs within strings adding particular underscores
	stepnumber=1,                    % the step between two line-numbers. If it's 1, each line will be numbered
	stringstyle=\color{stringcolor},     % string literal style
	tabsize=4,	                   % sets default tabsize to 2 spaces
	title=\lstname,                   % show the filename of files included with \lstinputlisting; also try caption instead of title
	inputencoding=utf8,
	literate=
	{á}{{\'a}}1 {é}{{\'e}}1 {í}{{\'i}}1 {ó}{{\'o}}1 {ú}{{\'u}}1
	{Á}{{\'A}}1 {É}{{\'E}}1 {Í}{{\'I}}1 {Ó}{{\'O}}1 {Ú}{{\'U}}1
	{à}{{\`a}}1 {è}{{\`e}}1 {ì}{{\`i}}1 {ò}{{\`o}}1 {ù}{{\`u}}1
	{À}{{\`A}}1 {È}{{\'E}}1 {Ì}{{\`I}}1 {Ò}{{\`O}}1 {Ù}{{\`U}}1
	{ä}{{\"a}}1 {ë}{{\"e}}1 {ï}{{\"i}}1 {ö}{{\"o}}1 {ü}{{\"u}}1
	{Ä}{{\"A}}1 {Ë}{{\"E}}1 {Ï}{{\"I}}1 {Ö}{{\"O}}1 {Ü}{{\"U}}1
	{â}{{\^a}}1 {ê}{{\^e}}1 {î}{{\^i}}1 {ô}{{\^o}}1 {û}{{\^u}}1
	{Â}{{\^A}}1 {Ê}{{\^E}}1 {Î}{{\^I}}1 {Ô}{{\^O}}1 {Û}{{\^U}}1
	{œ}{{\oe}}1 {Œ}{{\OE}}1 {æ}{{\ae}}1 {Æ}{{\AE}}1 {ß}{{\ss}}1
	{ű}{{\H{u}}}1 {Ű}{{\H{U}}}1 {ő}{{\H{o}}}1 {Ő}{{\H{O}}}1
	{ç}{{\c c}}1 {Ç}{{\c C}}1 {ø}{{\o}}1 {å}{{\r a}}1 {Å}{{\r A}}1
	{€}{{\euro}}1 {£}{{\pounds}}1 {«}{{\guillemotleft}}1
	{»}{{\guillemotright}}1 {ñ}{{\~n}}1 {Ñ}{{\~N}}1 {¿}{{?`}}1
}


% biblatex setup
% generally 9000 is ok, if higher than 10000 it's bad
% If you want to break on URL numbers
\setcounter{biburlnumpenalty}{9000}
% If you want to break on URL lower case letters
\setcounter{biburllcpenalty}{9000}
% If you want to break on URL UPPER CASE letters
\setcounter{biburlucpenalty}{9000}


\newcommand{\thesisuniversitylogo}{images/logo/polito_logo_2021_blu-2-2} % new PoliTo logo

% how to change Contents to Table of Contents
\addto\captionsenglish{% Replace "english" with the language you use
	\renewcommand{\contentsname}%
	{Table of Contents}%
}

% to change the name of Abbreviations to Acronyms
% not needed if use use entry types and define those
% \renewcommand{\abbreviationsname}{Acronyms}

% to allow line comments in algorithms
\algnewcommand{\LineComment}[1]{\State \(\triangleright\) #1}

% to declare abs and norm
\DeclarePairedDelimiter\abs{\lvert}{\rvert}%
\DeclarePairedDelimiter\norm{\lVert}{\rVert}%

% Swap the definition of \abs* and \norm*, so that \abs
% and \norm resizes the size of the brackets, and the 
% starred version does not.
\makeatletter
\let\oldabs\abs
\def\abs{\@ifstar{\oldabs}{\oldabs*}}
%
\let\oldnorm\norm
\def\norm{\@ifstar{\oldnorm}{\oldnorm*}}
\makeatother


% change this configuration with your info
\newcommand{\thesistitle}{Titolo della tesi}
\newcommand{\thesissubtitle}{Eventuale sottotitolo} % not strictly necessary, leave empty if not needed
\newcommand{\thesiscandidatename}{Name}
\newcommand{\thesiscandidatesurname}{Surname}
\newcommand{\thesissupervisoronename}{Name}
\newcommand{\thesissupervisoronesurname}{Surname}
\newcommand{\thesissupervisortwoname}{Name}
\newcommand{\thesissupervisortwosurname}{Surname}
\newcommand{\thesissupervisorthreename}{Name}
\newcommand{\thesissupervisorthreesurname}{Surname}
\newcommand{\thesisdate}{Mese Anno}
\newcommand{\thesiscourse}{Corso di Laurea}
\newcommand{\academicYear}{A.a. 202x/202x}
\newcommand{\thesisuniversity}{Politecnico di Torino}
\newcommand{\thesiscandidatetext}{Candidati}
\newcommand{\thesissupervisortext}{Relatori}


% fontsize is {size}{spacing}\family
\newcommand {\institutionfont}{\fontsize {20}{30}\scshape}
\newcommand {\divisionfont}{\fontsize {12}{20}\rmfamily}
\newcommand {\pretitlefont}{\fontsize {16}{16}\rmfamily}
\newcommand {\customtitlefont}{\fontsize {24}{28}\scshape}
\newcommand {\customsubtitlefont}{\fontsize {14}{28}\scshape}
% {iwona}{bx}{n}}
\newcommand {\fixednamesfont}{\fontsize {12}{16}\mdseries}
\newcommand {\namesfont}{\fontsize {12}{16}\mdseries}
\newcommand {\footfont}{\fontsize {15}{18}\rmfamily}

\onehalfspacing

\addbibresource{bibliography.bib}

% to load the glossaries, not needed if using bib2gls
% for glossary entry
% @entry{bird,
%     name={bird},
%     description = {feathered animal},
%     see={[see also]{duck,goose}}
% }

% if this bib file does not work, try using \input{file.tex}
% where all the \newabbreviation commands have been inserted
% containing all the definitions

% Gls to capitalize first letter
% GLS for full uppercase
% for abbreviations also
% glsxtrshort for abbreviation
% similar for long, full, and capital configurations, add pl at the end for plurals
% glsentryshort, long, plural (referred to shorts) must be used when in section titles
% glslink to allow the link but use a different text (as for href)


% if you want to use also description for the abbreviations/acronyms, you should use bib2gls and define all the entries in a bib file, which is incompatible with Overleaf
% EXAMPLES
\newacronym{DBMS}{DBMS}{Database Management System}
\makeglossaries

\begin{document}

\overfullrule=0.00001pt % latex shows a black bar for overfulls over this dimension

%\emergencystretch=1em % to allow some stretching in the lines to avoid overfull boxes, also in bibliography, eventually can be used only before bibliography or in the preamble for the whole document, not needed if using biblatex configuration in most cases


%% TOPTESI DEFAULT FRONTPAGE, IGNORED
%\ateneo{\thesisuniversity} % university name
%\logosede[5cm]{\thesisuniversitylogo} % logo, square brackets contain the height

%\titolo{\thesistitle} % title
%\sottotitolo{Metodo dei satelliti medicei} % subtitle

% place/remove a slash \\ to put the name on the following line or after Master Degree Course
%\corsodilaurea{\thesiscourse} % course name


%~251197 % id number is not needed

%\candidato{\thesiscandidatename~\textsc{\thesiscandidatesurname}} % candidate

% using tabular we can have more than 1 supervisor under the same column
%\relatore{\tabular{@{}l}%
%    \xmakefirstuc{\thesissupervisoronetitle}~\thesissupervisoronename~\textsc{\thesissupervisoronesurname}\\[0.4ex]
%    \xmakefirstuc{\thesissupervisortwotitle}~\thesissupervisortwoname~\textsc{\thesissupervisortwosurname}\\[0.4ex]
%    \xmakefirstuc{\thesissupervisorthreetitle}~\thesissupervisorthreename~\textsc{\thesissupervisorthreesurname}
%    \endtabular}
%\terzorelatore{Name}

% in this way we have Academic Year without stile=classica, so without lines
% \sedutadilaurea{\textsc{Academic~Year} 2019-2020}% per la laurea magistrale
% for PoliTo there is only month year
%\sedutadilaurea{\thesisdate}% per la laurea magistrale
% PhD
%\esamedidottorato{Novembre 1610}
%\ciclodidottorato{XV}

% offset for binding, the smaller the better
%\setbindingcorrection{3mm}


%\iflanguage{english}{%
	%\retrofrontespizio{This work is subject to the Creative Commons Licence}

%	\CorsoDiLaureaIn{\thesislevel's Degree Course in\space}

%	\TesiDiLaurea{\thesislevel's Degree Thesis}

%	\InName{in}
%	\CandidateName{\xmakefirstuc{\thesiscandidatetext}}% or Candidates
%	\AdvisorName{\xmakefirstuc{\thesissupervisortext}}% or Supervisor
	%\TutorName{Tutor}
	%\NomeTutoreAziendale{Internship Tutor}

	%\NomePrimoTomo{First volume}
	%\NomeSecondoTomo{Second Volume}
	%\NomeTerzoTomo{Third Volume}
	%\NomeQuartoTomo{Fourth Volume}
%}{}

\lstdefinelanguage{JavaScript}{
  keywords={break, case, catch, continue, debugger, default, delete, do, else, finally, for, function, if, in, instanceof, new, return, switch, this, throw, try, typeof, var, void, while, with, const, import, async, await},
  morecomment=[l]{//},
  morecomment=[s]{/*}{*/},
  morestring=[b]',
  morestring=[b]",
  sensitive=true
}

\lstset{
   language=JavaScript,
   extendedchars=true,
   basicstyle=\footnotesize\ttfamily,
   showstringspaces=false,
   showspaces=false,
   numbers=left,
   numberstyle=\footnotesize,
   numbersep=9pt,
   tabsize=2,
   breaklines=true,
   showtabs=false,
   captionpos=b
}


\english % uncomment for English

% front page -- add your info in common/thesis_info.tex
% EDIT THE thesis_info.tex FILE, NOT THIS ONE!
\newgeometry{top=4cm,left=3cm,right=3cm,bottom=4cm,heightrounded}
\begin{titlepage}
\centering
%
\includegraphics[width=110mm]{\thesisuniversitylogo}\\
%
\vspace{\stretch{2}}
%
{\institutionfont \textbf{\thesisuniversity} \par}
%
\vspace{\stretch{1}} % changing this number and the others changes the proportion
%
{\divisionfont \thesiscourse \par}
{\divisionfont \academicYear \par}
\iflanguage{english}{
    {\divisionfont Graduation Session \thesisdate \par}
}{\divisionfont Sessione di laurea \thesisdate \par}
%

\vspace{\stretch{4}}
%
{\customtitlefont \textbf{\thesistitle} \par}
{\customsubtitlefont \textbf{\thesissubtitle} \par}
%
\vspace{\stretch{4}}
%
\makebox[\textwidth]{\null\hfill\def\arraystretch{1.1} % to change the spacing change this number
\begin{minipage}[t]{.4\textwidth}
    \begin{tabular}[t]{@{}ll@{}}
        \fixednamesfont \thesissupervisortext: & \\
         & \namesfont \thesissupervisoronename~\thesissupervisoronesurname \\
         & \namesfont \thesissupervisortwoname~\thesissupervisortwosurname %\\
         % uncomment if you need a third supervisor
         % & \namesfont \thesissupervisorthreename~\thesissupervisorthreesurname
    \end{tabular}
\end{minipage}
%
\hfill
%
\begin{minipage}[t]{.4\textwidth}
\begin{tabular}[t]{@{}ll@{}}
    \fixednamesfont \thesiscandidatetext: & \\
     & \namesfont \thesiscandidatename~\thesiscandidatesurname
\end{tabular}
\end{minipage}\hfill\null}\\
%
\vspace{\stretch{1}}
%
\end{titlepage}

\restoregeometry
 % new frontespizio (2023)
%\retrofrontespizio
% insert text for the back of the front page
% if you insert any remove the following \paginavuota
% either a blank page or a back is needed to have double-sided printing
% pay attention to leave the space for the page

%\paginavuota % clears a page

\frontmatter % not strictly needed, as toptesi takes care of it already

% abstract if needed
% \begin{abstract}
%     % abstract, choose between abstract and summary

For the better understanding of what happens during a visual experiment, the possibility of recording the experiment itself and watching it again at a later time is crucial.
While for many cases achieving this is just a matter of recording the event with a camera, some complex situations require more attention.
One such case is when the nature of the experiments is to understand 3D data, where a 2D video would not carry the necessary information.

This thesis aims at recording 3D trajectories of small bubbles moving in the air.
This is achieved by combining the information provided by a set of synchronized cameras, observing the experiment.
The source videos capture the same scene from different positions, which are known thanks to an initial calibration process.
The special-purpose cameras are able to provide binary frames, highlighting the bubbles in white, on top of a black background.
The proposed solution starts by transforming each frame into a list of 2D coordinates, describing where the center of each bubble is within the image.
For each time instant, the different points of view are then leveraged to reproject the coordinates into the 3D space, creating a cloud of 3D points.
Subsequently, the intrinsic sequential nature of the video is made explicit: reconstructed bubbles of consecutive frames are joined together, to form trajectories.
Finally, the obtained 3D reconstruction needs to be displayed on a 2D monitor: different techniques are exploited to make this visualization as intuitive and understandable as possible.

This thesis illustrates all the approaches evaluated for each step, and which ones are eventually chosen thanks to their speed and quality performance.
The final solution is a pipeline that can process in real-time the experimental setup, displaying the reconstructed trajectories only a few seconds after the image capture.

% \end{abstract}

% to create blank pages for openright in frontmatter
% use one of the following two methods
% 1) use the following three lines
%\phantom{0} % needed otherwise cleardoublepage does not clean the page because it sees it empty
%\cleardoublepage
%\thispagestyle{empty} % to have empty page, without numbers
% 2) or
\paginavuota % to manually create a blank page

% summary (uncomment until "end summary" if needed)
%\sommario
%% only the text for the summary

Lorem ipsum dolor sit amet, consectetur adipiscing elit, sed do eiusmod tempor incididunt ut labore et dolore magna aliqua. Sed arcu non odio euismod lacinia at quis risus sed. Tempor commodo ullamcorper a lacus vestibulum. Sed elementum tempus egestas sed sed risus. Netus et malesuada fames ac turpis. Sed blandit libero volutpat sed cras ornare arcu dui vivamus. Vel pharetra vel turpis nunc eget lorem dolor. Iaculis nunc sed augue lacus viverra vitae congue. Orci eu lobortis elementum nibh. Faucibus nisl tincidunt eget nullam non nisi est sit. Ante metus dictum at tempor commodo ullamcorper. Tempus quam pellentesque nec nam. Nulla malesuada pellentesque elit eget gravida.


% $400\times$ is nicer than 400x


%\phantom{0}
%\cleardoublepage
%\thispagestyle{empty}
% end summary

\ringraziamenti % acknowledgements
\newcommand{\ackspace}{\vspace{0.3cm}}
\newcommand{\ackspacebullet}{}

The writing of this thesis marks the end of a large chapter in my life, a chapter filled with intense learning moments, interleaved with lighter, funnier times, as well as moments of deep reflection that enabled me to grow as a person.
As last words of this life chapter, I deeply wish to thank from the bottom of my hearth all the people around me that shaped and made possible these last years:\ackspacebullet
\begin{itemize}
	\itemsep 0em
	\item I wish to thank all the numerous professors that dedicated their time and effort to transmit a fraction of their extensive knowledge to me.
	      Thanks for all the passion that you dedicate every day to form the generations that will shape the future.\ackspacebullet
	\item I wish to thank professor Cantoro for being my thesis supervisor.
	      Despite the thesis not ending up in your research interest, your presence was still extremely valuable in directing the overall structure of this paper.
	      On top of that, the knowledge you passed me with your GPU Programming course has been extremely versatile while developing the different approaches described in this document.\ackspacebullet
	\item I wish to thank my family, mostly my parents Fulvio and Daniela and my sister Valentina, for your continuous support in both moral and economical term.
	      Thanks for the way you grew me up, thanks for always believing in me, for being there when I needed support.
	      Thanks for pushing me to go forward even when my decisions were not matching your expectations.\ackspacebullet
	\item I wish to thank my grandparents Franco, Luciana, Domenico and Secondina, and all my relatives in general.
	      You always trusted and encouraged me to be the best version of myself, teaching me with your example.\ackspacebullet
	\item I wish to thank Samuele, my travel companion in Turin for all these years.
	      Thanks for inviting me to rent a room in your apartment: that house and the memories I have in it will forever live in my hearth.
	      Thanks for all the train trips we had together, the meals and cooking times we shared, the games we played together.\ackspacebullet
	\item I wish to thank my university friends Davide, Federico and Giacomo.
	      The university years are not only a time to get filled of knowledge, but mostly they aim to expand one's relationships, interests and horizons.
	      With all your chatting, suggestions, quirky behaviors and ideas, you sculpted a great portion of how I am right now.\ackspacebullet
	\item I wish to thank Kamel and François for all the practical suggestions that you gave me during my stay at SMA-RTY.
	      Thanks for doing whatever was in your power to make me feel welcomed, not only in the office, but in Clermont in general.\ackspacebullet
	\item I wish to thank John and Yorick, not only you were office colleagues, you became real friends for me.
	      Thanks for all the lunches and coffee-breaks we had together, for all the discussions and shared knowledge.\ackspacebullet
	\item I wish to thank Edoardo for being my \textit{Italian of reference} in the office.
	      Thanks for the discussions about the problems I was encountering: talking with someone was always a good way to find new creative solutions.
	      Thanks also for the more relaxed moments: it was a relief to have someone to whom I was able to speak Italian, to relax my brain from all the English and French I constantly had to speak.\ackspacebullet
	\item I wish to thank Emmanuel and Emilie, Anne and Louis, Jérôme and Laurence, and all the people I met at the church in Aubière.
	      Thanks for the warm welcome to the parish, thanks for inviting me to countless \textit{``repas''} and for asking me to play the guitar at the Mass.
	      You really filled my French weekends with joy, and allowed me to realize how we are all brothers in Christ, regardless of where we come from.\ackspacebullet
	\item I wish to thank my friends from Manta: Marta, Martina, Martina, Nicolò, Samuele, Simone and Stefano.
	      While not being directly present in my university career, you played a crucial role in creating lovely moments with me during the weekends.
	      Thanks for the countless dinners, games and chats we had together.\ackspacebullet
	\item Finally, I wish to thank all the people that are not in this list.
	      I strongly believe that we are what we are thanks to the sum of all the single events that happened in our lives.
	      I think that every action, every person, every word we experienced contributed to sculpt what we are today.
	      As such, I deeply desire to thank everyone that I met on my path, may it be \textit{for the life, or just a day}, as an Italian song says: thanks for making me as I am now, which I am really proud of.\ackspacebullet
\end{itemize}
As the people close to me probably already know, computer engineering was a deep passion of mine, but not the dream of my life.
After closing this chapter, I will soon open a new one, crowning the dream I had since I was a child.
I will be in Bremen, studying to become an airline pilot at the Lufthansa group flight school.\ackspace

I am certain that all the experiences lived this past chapter of my life will be of great support in this new adventure: once more, I deeply wish to thank everyone that contributed to it.
Finally, I would like to ask you a favor: as much as you supported me in this past few years, please continue to do so in the upcoming adventure.
It won't be easy, it won't be possible with the same modalities... but I really need your support.\ackspace

Thanks to everyone, \\
\null\hfill Francesco


\paginavuota
\tableofcontents

%\listoftables % ToC for tables

\listoffigures % ToC for figures

% actually abbreviation is the name used for acronym in glossaries-extra
% title sets the name
% type tells the type of glossary to print
% style overrides the global style
% here we are printing only abbreviations
% printunsrtglossary if using record, otherwise printglossary is ok
%\paginavuota
%\printunsrtglossary[style=altlist,title=Glossario,type=\glsxtrabbrvtype]

% also list of symbols here if needed

% to remove all first use occurrences given the presence of the summary
\glsresetall
% to skip all the first use occurrences, using only short forms
% \glsunsetall


\mainmatter

%\part{Prima Parte} % parts division, not needed
% Chapters always open on a right-side page, i.e. odd numbers, so a blank page is inserted if needed
%\cleardoublepage[empty] % to have a fully blank page
% a blank page appears before the first chapter in some configurations, on the last version it doesn't

% chapters go here
\chapter{Introduzione}
\label{sec:introduction}

Lorem ipsum dolor sit amet, consectetur adipiscing elit, sed do eiusmod tempor incididunt ut labore et dolore magna aliqua. Porttitor eget dolor morbi non arcu risus quis varius. Libero id faucibus nisl tincidunt. Neque laoreet suspendisse interdum consectetur libero id faucibus nisl tincidunt. Scelerisque in dictum non consectetur a erat. Leo a diam sollicitudin tempor id eu. Sodales ut eu sem integer vitae justo eget magna fermentum. A cras semper auctor neque vitae. Cursus euismod quis viverra nibh cras pulvinar. Mi tempus imperdiet nulla malesuada pellentesque elit eget gravida cum. Dictum at tempor commodo ullamcorper a lacus vestibulum sed. Ultricies mi eget mauris pharetra. In pellentesque massa placerat duis ultricies lacus. Fringilla phasellus faucibus scelerisque eleifend donec pretium vulputate.

Aliquam sem fringilla ut morbi tincidunt augue interdum. Risus at ultrices mi tempus imperdiet nulla malesuada pellentesque. Semper quis lectus nulla at volutpat. Nullam non nisi est sit amet facilisis. Eget velit aliquet sagittis id consectetur purus ut faucibus pulvinar. Ultricies tristique nulla aliquet enim tortor at auctor urna nunc. Pharetra diam sit amet nisl suscipit adipiscing bibendum est ultricies. Amet facilisis magna etiam tempor. Nunc non blandit massa enim nec dui nunc mattis. At ultrices mi tempus imperdiet nulla malesuada pellentesque. Cursus metus aliquam eleifend mi in nulla posuere. In eu mi bibendum neque egestas congue quisque. Augue eget arcu dictum varius duis at consectetur.

% use [] to set name for ToC
\section[Goal]{Goal} % ok with fontsize=12pt

Lorem ipsum dolor sit amet, consectetur adipiscing elit, sed do eiusmod tempor incididunt ut labore et dolore magna aliqua. Tristique senectus et netus et malesuada fames ac turpis. Pretium nibh ipsum consequat nisl vel pretium lectus quam. Urna molestie at elementum eu facilisis sed odio morbi quis. Sed egestas egestas fringilla phasellus faucibus scelerisque eleifend. At in tellus integer feugiat. Mauris rhoncus aenean vel elit scelerisque mauris pellentesque pulvinar. Commodo sed egestas egestas fringilla. Nunc lobortis mattis aliquam faucibus purus in massa. Facilisis magna etiam tempor orci eu lobortis elementum nibh. Elementum curabitur vitae nunc sed velit dignissim. Neque volutpat ac tincidunt vitae. Massa id neque aliquam vestibulum morbi blandit cursus risus at. Porta non pulvinar neque laoreet suspendisse interdum consectetur. Turpis in eu mi bibendum. Ut tristique et egestas quis ipsum suspendisse. Integer quis auctor elit sed vulputate mi sit amet. Viverra nam libero justo laoreet sit amet cursus.

\section[Struttura della tesi]{Struttura della tesi} % ok with fontsize=12pt

Lorem ipsum dolor sit amet:
\begin{itemize}
    \item onsectetur adipiscing elit,
    \item sed do eiusmod tempor incididunt ut labore et dolore magna aliqua.
    \item Tristique senectus et netus et malesuada fames ac turpis.
\end{itemize}
%\input{content/chapters/chapter2}
%\input{content/chapters/chapter3}
%\input{content/chapters/chapter4}
%\input{content/chapters/chapter5}
%\input{content/chapters/chapter6}

% \paginavuota % it works even without stile=classica

%\appendix
%% appendix
\chapter{Galileo}
\label{sec:appendix_galileo}

\lstdefinelanguage{JavaScript}{
	keywords={break, case, catch, continue, debugger, default, delete, do, else, finally, for, function, if, in, const, instanceof, new, return, switch, this, throw, try, typeof, var, void, while, with},
	morecomment=[l]{//},
	morecomment=[s]{/*}{*/},
	morestring=[b]',
	morestring=[b]",
	sensitive=true
}

%\lstinputlisting[]{} % for source code files directly
% lstlisting environment for direct inclusion
\begin{lstlisting}[language=JavaScript]
    const test = test;
\end{lstlisting}

% for computational complexity
$\mathcal{O}\left(n\log{n}\right)$

% verbatim
\verb+numpy+



% endnotes here if needed

\phantom{0}
\cleardoublepage
\printbibliography[heading=bibintoc] % heading required to show it in ToC

\end{document}
