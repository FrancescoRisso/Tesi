\subsection{Torch.unfold concept}
\label{sec:locate:torchunfold}

The \texttt{unfold} function from \texttt{PyTorch} takes an image, and breaks it into either disjoint or partially overlapping tiles.
The idea is to use a divide and conquer approach, where the full image is split into smaller images, hopefully making it faster.


\subsubsection{Algorithm}

The overall algorithm divides the image into patches, to then process each one separately.
Different tile sizes were compared, to find the best, if any.

\subsubsection{Evaluation}

As visible in table~\ref{tab:torch.unfold}, having smaller patches increases the time required to perform the overall \textit{Locate}.
This is likely due to the overlap between patches.
The overlap is however necessary, to avoid bubbles split across patches to be considered as separate.

\begin{table}[ht]
	\centering
	\def\arraystretch{2}
	\begin{tabularx}{\linewidth}{
		|>{\arraybackslash}p{.2\linewidth}
		|>{\centering\arraybackslash}X
		|>{\centering\arraybackslash}X
		|>{\centering\arraybackslash}X
		|>{\centering\arraybackslash}X
		|>{\centering\arraybackslash}X
		|>{\centering\arraybackslash}X|
		}
		\hline
		\textbf{Patch size [px]} & 501{$\times$}501  & 101{$\times$}101  & 51{$\times$}51   & 25{$\times$}25    & 15{$\times$}15    & 11{$\times$}11    \\ \hline
		\textbf{Time [s]}         & 1.57 & 3.90 & 7.02 & 18.22 & 49.65 & 96.59 \\ \hline
	\end{tabularx}
	\def\arraystretch{1}
	\caption{Time required to process 1 frame with different patch sizes}
	\label{tab:torch.unfold}
\end{table}
