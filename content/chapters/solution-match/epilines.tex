\subsection{Closest to epiline}
\label{sec:match:epiline}

As described in section~\ref{sec:backgr:stereo:epilines}, a spcecific point in a camera will be seen on a specific epiline on another camera.

In traditional stereoscopy, the epiline is a set of discrete pixels, while in our case it's a continuous line of floating-point values.
On top of that, the calibration process may have some imperfections, that lead the epiline to be potentially slightly wrong.
Due to these two factors, it is impossible to have a bubble perfectly on top of the epiline: instead, the match is selected as the closest bubble to the epiline, provided that the distance is under a reasonable threshold.

This is one of the approaches originally used in the Matlab script provided by the research group.
It was considered as a candidate algorithm, and as such it was re-implemented in Python for easier comparison.

\subsubsection{Algorithm}

Chosen a ``main'' camera, the algorithm processes each camera with the following steps:
\begin{enumerate}
	\itemsep 0em
	\item For each bubble in the main camera:
	      \begin{enumerate}
		      \item Compute the coefficients $a$, $b$ and $c$ of the epiline $ax+by+c{=}0$;
		      \item Compute the distance (with a scale factor) of all bubbles $(x_i, y_i)$ in the side camera to the epiline: $d = ax_i + by_i + c$;
		      \item Select as candidate match the bubble with $d=min_i(d_i)$;
		      \item Compare $d$ with the reasonability threshold $T$:
		            \begin{itemize}
			            \item If $d<T$, consider the bubble as a valid match;
			            \item Otherwhise, consider the original bubble as unmatched.
		            \end{itemize}
	      \end{enumerate}
\end{enumerate}

\subsubsection{Evaluation}

The algorithm had excellent speed, running at 300 FPS, which is 10x faster than the requirements.
While the quality seems quite good with the smaller dataset (with 28/0/2 correct/missed/wrong matches), the performance is much worse with more bubbles (51/4/45).
As such, there is space for improvement, trading some of the useless speed with better quality.
