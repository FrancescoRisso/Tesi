\subsection{Epilines + median correction}
\label{sec:match:epi-median}

As previously stated, the offset between two camera views of the same bubble is consistent across all the bubbles in the scene.
This fact is used in this approach to find and potentially correct mistakes.
Originally, the average displacement was considered as comparison term, but it would be very sensitive to errors.
Instead, the median displacement is computed, as a vector whose components are the medians of the displacement componets.
As an example, displacements of $(10,)$, $(11,0)$ and $(-3, -1)$ would generate a median displacement of $(10, 0)$.

\subsubsection{Algorithm}

For each bubble in the main camera, each other camera is processed with the following steps:
\begin{enumerate}
	\itemsep 0em
	\item For each bubble in the main frame:
	      \begin{enumerate}
		      \item Compute the coefficients $a$, $b$ and $c$ of the epiline $ax+by+c{=}0$;
		      \item Compute the distance (with a scale factor) of all bubbles $(x_i, y_i)$ in the side camera to the epiline: $d_i = ax_i + by_i + c$;
		      \item Select as candidate match the bubble with $d=min_i(d_i)$;
		      \item Compare $d$ with the reasonability threshold $T$:
		            \begin{itemize}
			            \item If $d<T$, consider the bubble as a valid (temporary) match;
			            \item Otherwhise, consider the original bubble as unmatched.
		            \end{itemize}
	      \end{enumerate}
	\item Compute the median of all the matches of the frame;
	\item For each bubble in the main frame:
	      \begin{enumerate}
		      \item Compute the displacement and compare it with the median:
		            \begin{itemize}
			            \item If both the difference in length and the angle between the vectors are under specific thresholds, confirm the match, and continue with the next bubble;
			            \item Otherwhise, correct the match by performing the following steps;
		            \end{itemize}
		      \item With a procedure similar to steps 1.a to 1.d, find the $N$ (parameter) bubbles closest to the epiline;
		      \item Among them, find the one whose displacement is the most similar to the median displacement;
		      \item Check the correctness of this new match:
		      \begin{itemize}
				\item If the bubble is further than the threshold $T$ (step 1.d) from the epiline, consider the match wrong and remove it;
				\item If the new displacement still does not satisfy the distance and angle from the median, consider the match wrong and remove it;
				\item Otherwhise, correct the previous match with this one.
			  \end{itemize}
	      \end{enumerate}
\end{enumerate}

\subsubsection{Evaluation}

This algorithm runs at 55 FPS, while improving on the quality: for the 30-bubbles dataset the distribution of correct/missed/wrong matches is 27/2/1, while for the 100-bubbles dataset it is 74/20/6.
