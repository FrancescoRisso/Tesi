\subsection{Epilines + brute force}
\label{sec:match:epi-bruteforce}

Given the knowledge that the correct match is near the epiline, there is no need to perform a brute force check on all the bubbles, as proposed in the previous approach.
Instead, it is enough to check the bubbles close to the epiline.

\subsubsection{Algorithm}

\begin{enumerate}
	\itemsep 0em
	\item For each bubble in the main camera:
	      \begin{enumerate}
		      \item Compute the distance $d$ (with a scale factor) of all bubbles in each side camera from the corresponding epiline;
		      \item Consider only the bubbles with $d<T$, for a specific threshold $T$ (if there are none, leave the bubble unmatched);
		      \item Reconstruct the 3D position, matching it with all the bubbles on one non-main camera. The result will be a set of 3D points ($P_i$);
		      \item Do the same, with the other non-main camera (to obtain a set $P_j'$);
		      \item Find the points $p\in P_i$ and $p'\in P_j'$ whose distance is the smallest;
		      \item Check their distance:
		            \begin{itemize}
			            \item If it is below a specific threshold, the two reconstructions are considered to be the same bubble, hence the two matches are approved;
			            \item Otherwise, the reconstructed bubbles are the closest plausible, but still different bubbles, hence the match is considered missing.
		            \end{itemize}
	      \end{enumerate}
\end{enumerate}

\subsubsection{Evaluation}

This approach yields excellent result, with no mistakes recorded.
This is achieved thanks to its requirement to have a ``3-way confirmation'' on the bubbles: when in doubt, it prefers to leave the bubble unmatched.
The qualitative results were 18/12/0 correct/missing/wrong matches on the 30-bubbles frame and 55/45/0 on the 100-bubbles dataset.
This however comes at a slight cost of speed, since this algorithm is only able to reach 19 FPS.
