\chapter{The \linkDD* step}
\label{chap:2dlink}

The \link* step aims to link together consecutive time instants, by joining together the coordinates of each individual bubble across the varius time instants it is seen.
The result is a series of \textbf{tracklets}.
Specifically, the 2D version of the \link* step operates on 2D coordinates, producing 2D tracklets.

\section{State of the art}

For the \link* step, the literature research was less successful: no new approach was found, and only some of the libraries found for the \locate* step were also performing the task:
\begin{itemize}
	\itemsep 0em
	\item Section~\ref{sec:link2d:trackpy} explores the Trackpy~\cite{trackpy} Python library;
	\item Section~\ref{sec:link2d:myptv} explores the MyPTV~\cite{myptv} Python library.
\end{itemize}

\section{Requirements}

\subsection{Input}

The input to the \link* step coincides with the output of the previous step.
In the pipeline is using \linkDD*, the full pipeline under exam is \locate* - \link* - \match* - \visual*.
This means that the input to this step is the output of the \locate* step, as described in section~\ref{sec:locate:output}.

\subsection{Output}

The coordinates of the particles in the tracklets follow the same format as the \locate* output.
A four-dimensional \texttt{positions} array describes the \texttt{(x, y)} coordinates of the bubbles inside \texttt{positions[C][F][B]}, $C$, $F$ and $B$ being the camera, frame and bubble indices, respectively.
The difference with the \locate* format is that values of $B$ are scoped across the whole acquisition, not limited to the single frame.
This means that all values with the same $C$ and $B$ will represent the same real bubble across the different frames.

With this representation, valid bubbles are not clusterized at the smallest values of $B$: for example, bubble $B{=}0$ may disappear after some frames, leaving the rest of its tracklet to contain invalid positions.
As such, the \texttt{numTracers} array is not anymore enough to describe the valid positions.
Instead, a different array is used: \texttt{validTracers} is a three-dimensional, boolean array.
\texttt{validTracers[C][F][B]} contains the information of whether the bubble $B$ of camera $C$ was detected at frame $F$.
False values indicate that, at a specific frame, the bubble was either not found yet, or already lost, or the overall number of bubbles traced by camera $C$ was lower than $B$.

\subsection{Speed}

As for the \locate* step, each second the \linkDD* has to process the bubbles from $N{\cdot}f$ frames, where $N$ is the number of cameras and $f$ is their framerate.
As such, the required speed for this step is the same 90 FPS that is required by the \locate* step.

\subsection{Quality}

The overall quality of an algorithm can be estimated by combining manual observation with the number of resulting tracklets found.
For the manual observation, the input video was overlayed with a tail composed of points and segments, describing the last few frames of trajectory.
Figures~TODO are examples of frames used for manual observation: the single links are quite small, it is hard to see them individually, it's much easier to consider the general view.

Possible situations of reconstructed links are:
\begin{itemize}
	\itemsep 0em
	\item Link correctly detected: the number of total tracklets does not change from the previous frame, and the link is coherent with the rest of the trajectory;
	\item Link not detected: visually, it's hard to notice the missing link; however, this splits the tracklet into two pieces, increasing the number of tracklets by 1;
	\item Wrong link detected: the number of tracklets remains the same, while an inconsistent movement is visible by eye.
\end{itemize}
As such, a good reconstrucion is one with few tracklets and a coherent visual representation.

\section{Approaches}

The following sections describe the different approaches evaluated for the \linkDD* step.
They are evaluated on a 201-frames video~\cite{linkDD-original}, whose frames look like figure~\ref{fig:locate:original}.
For the different approaches, a crop of a sample frame is reported as per the \locate* approaches (section~\ref{sec:locate:approaches}), with the tail of the tracklet.
Full videos are available on YouTube, following the links in the corresponding citations.

\newpage
\subsection{Trackpy}
\label{sec:link2d:trackpy}

The \texttt{link} function from the Trackpy~\cite{trackpy} library is able to perform the \link* task both in 2D and in 3D.

Originally, it required the located positions to be inside a \texttt{Pandas Dataframe}, and it used to convert it into a \texttt{NumPy} array.
However, since our data was already inside a \texttt{NumPy} array with the same format, the library was altered to avoid this useless conversion, thus saving time.

\subsubsection{Algorithm}

The Trackpy library implements the Crocker-Grier linking algorithm~\cite{trackpy-link}.

\subsubsection{Evaluation}

For a single camera, the linking speed was 40 FPS.
If different cameras were analyzed in parallel processes, 3 cameras could be processed at an overall speed of 120 FPS.

The quality was good at a visual inspection (see figure~\ref{fig:linkDD:trackpy} or the full video~\cite{linkDD-trackpy}), and the total number of tracklets was around 6500.

\begin{figure}
	\centerline{\includegraphics[width=\locateimgsize]{images/link2d/trackpy.png}}
	\caption{\centering A frame from the Trackpy \linkDD* result, full video available at~\cite{linkDD-trackpy}}
	\label{fig:linkDD:trackpy}
\end{figure}
 \newpage
\subsection{MyPTV}
\label{sec:link2d:myptv}

MyPTV~\cite{myptv} implements \linkDD* with the \texttt{2D\_tracking} command.

\subsubsection{Algorithm}

The library implements the ``best estimate method'' described by Ouellette-Xu~\cite{linkDD-myptv}: a nearest neighbor initial guess, followed by 4-frame tracking for the following frames.

\subsubsection{Evaluation}

The speed of this approach, lower than 1 FPS, was unacceptable: as such, the quality was not even evaluated.
 \newpage
\subsection{Kalman CPU}
\label{sec:link2d:kalman-cpu}

The iterative concept for the \locate* approach (described in section~\ref{sec:locate:iterative}) was using the trajectories to predict the future position of each bubble.
As such, it was already performing the \linkDD* task.
For this reason, a modified version was considered as a novel \linkDD* approach.

\subsubsection{Algorithm}

The algorithm starts with an empty list of previously found bubbles.
It then loops over these steps, for each frame in the video:
\begin{enumerate}
	\itemsep 0em
	\item Store the positions of the located bubbles in a suitable data structure. Based on the settings, a basic coordinate list or a 2D binary tree could be used;
	\item For each bubble in the ``previous bubbles'' list:
	      \begin{enumerate}
		      \item Compute its velocity and acceleration from the last trajectory points, if available;
		      \item Compute a predicted position;
		      \item Find the candidate bubbles in the next frame:
		            \begin{itemize}
			            \item If the binary tree was used as representation, consider all bubbles within a \textit{delta} from the predicted position;
			            \item If the bubbles list was used as representation, consider all bubbles;
		            \end{itemize}
		      \item Among the candidates, chose the closest one (in terms of Manhattan distance) to the predicted position;
		      \item Check that the distance of the match is reasonable:
		            \begin{itemize}
			            \item If it is, link the two bubbles, and mark the chosen one as linked;
			            \item If it is not, consider the bubble lost for this frame. If the bubble is lost for multiple consecutive frames, it is removed from the list;
		            \end{itemize}
	      \end{enumerate}
	\item Add all unlinked bubbles to the list, as new tracklets.
\end{enumerate}
The 2D binary tree was added to reduce the amount of possible matches, by splitting the coordinates into 4 bins for every node layer, storing the bins as Python lists.
To obtain the bubbles within a \textit{delta} from the predicted position, the tree would consider all bins that (at least partially) satisfied the condition.

\subsubsection{Evaluation}

Considering the tree choice, it was possible to choose between:
\begin{itemize}
	\itemsep 0em
	\item no tree at all: as stated in the algorithm overview, the original \texttt{NumPy} array of bubbles was used as set of candidates;
	\item tree with 0 layers: no split was performed, therefore the original \texttt{NumPy} array was simply translated into a Python list. The distance computation is the same as with no tree, with added overhead of creating and accessing the Python list instead of the \texttt{NumPy} array.
	\item tree with maximum number of layers: each leaf bin represents a single pixel of the original image, therefore will contain either 1 or 0 bubbles. The number of layers is $\lceil log_2(P) \rceil$, where $P$ is the side, in pixels, of the captured image. In our case, $P{=}960$, prompting to choose 9 layers. This option has the most fine-grained way to choose the bubbles at a specific distance. It however adds the most overhead related both to constructing a bigger tree, and traversing multiple paths to construct the set of bubbles to be evaluated.
	\item tree with intermediate number of layers: this is a compromise between efficiency in building and using the tree, and reducing the number of distances to compute.
\end{itemize}
These different approaches are compared in figure~\ref{fig:linkDD:kalmancpu:speed-cmp}: when considering the tree, the best choice is a compromise between granularity and tree complexity.
However, the overhead of building the tree does still not match the performance without it, thanks to the extreme optimization of the \texttt{NumPy} library.
As such, the version with no tree was the chosen one.

\begin{figure}
	\centerline{\includegraphics[width=\locateimgsize]{images/link-2dtree-speeds.png}}
	\caption{\centering Comparing speeds for different tree sizes in the Kalman CPU \linkDD* approach}
	\label{fig:linkDD:kalmancpu:speed-cmp}
\end{figure}

As visible in figure~\ref{fig:linkDD:kalmancpu:speed-cmp}, the overall speed is limited to about 14 FPS, much slower than Trackpy.
On top of that, the result is also slightly worse: while it looks good at visual inspection (see figure~\ref{fig:linkDD:kalmancpu}), the total number of traces is higher than Trackpy, at about 8400 tracklets found.

\begin{figure}[H]
	\centerline{\includegraphics[width=\locateimgsize]{images/link2d/kalman_CPU.png}}
	\caption{\centering A frame from the Kalman CPU \linkDD* result, full video available at~\cite{linkDD-kalman-cpu}}
	\label{fig:linkDD:kalmancpu}
\end{figure}
 \newpage
\subsection{Kalman GPU}
\label{sec:link2d:kalman-gpu}

...

\subsubsection{Algorithm}

...

\subsubsection{Evaluation}

...

% \begin{figure}
% 	\centerline{\includegraphics[width=\locateimgsize]{images/link/...}}
% 	\caption{\centering ...'s result}
% 	\label{fig:locate:...}
% \end{figure}
 \newpage

\section{Final choice}

Figure~\ref{fig:linkDD:speed} compares the speeds of the various \linkDD* approaches: Trackpy is the only one with an adequate speed.
On top of that, it is also the approach with the best overall quality.
As such, if the pipeline is traversed in the \locate* - \link* - \match* - \visual* order, the \link* step will use the Trackpy implementation.

\begin{figure}
	\centerline{\includegraphics[width=\locateimgsize]{images/link2d-speed-comparison.png}}
	\caption{\centering Comparing the speeds of the different \linkDD* approaches}
	\label{fig:linkDD:speed}
\end{figure}
