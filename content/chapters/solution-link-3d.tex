\chapter{The \linkDDD* step}
\label{chap:3dlink}

The \link* step aims to link together consecutive time instants, by joining the coordinates of each individual bubble across the varius time instants it is seen.
The result is a series of \textbf{tracklets}.
Specifically, the 3D version of the \link* step operates on 3D coordinates, producing 3D tracklets.

\section{Requirements}

\subsection{Input}

The data for this pipeline step comes from the output of the \match* step, described in section~\ref{sec:match:output}.
The input is therefore composed of two arrays, \texttt{positions} and \texttt{validTracers}.

\subsection{Output}

Similarly to the \linkDD* step (whose ouptut is described in section~\ref{sec:linkDD:output}), the output format is the same as the input, with the added constraint that equal bubble indices across different framess imply same real-life bubble.

The output will be therefore composed of the three-dimensional, floating-point \texttt{positions[F][B]} array, with the \texttt{(x, y, z)} coordinates of bubble $B$ at time $F$, and the corresponding two-dimensional, boolean validity array \texttt{validTracers}.

\subsection{Speed}

Working directly on 3D data and not on the single cameras, the \linkDDD* only needs a speed of 30 FPS, similar to the \match* step.

\subsection{Quality}

Similarly to the \linkDD* step, the quality can be estimated by number of tracklets and visual inspection.

\section{State of the art}

Similarly to the previous steps, further research did not find more tools for performing the \linkDDD* task.
Among the ones already found, only Trackpy~\cite{trackpy} was able to perform \linkDDD*.
Its algorithm and performance are evaluated in section~\ref{sec:link3d:trackpy}.

\section{Approaches}

In the following sections, the two approaches to \linkDDD* are explained.
They are evaluated on the same 201-frames dataset used for the \linkDD* step, preprocessed by the \match* step.

\newpage
\subsection{Trackpy}
\label{sec:link3d:trackpy}

With minor modifications, it was possible to update the Trackpy \linkDD* (described in section~\ref{sec:link2d:trackpy}) for performing \linkDDD*.

\subsubsection{Algorithm}

The Trackpy library implements the Crocker-Grier linking algorithm~\cite{trackpy-link}.

\subsubsection{Evaluation}

The speed is similar to the Trackpy \linkDD* for one camera, at 38 FPS.
However, most of the tracklets were extremely short, resulting in about 9300 individual tracklet.
 \newpage
\subsection{Nearest neighbor}
\label{sec:link3d:NN}

Moving from 2D space to 3D space, the bubbles are naturally much sparser.
As such, the 3D bubbles are far enough away from each other, that the nearest neighbor is undoubtedly the correct link.

\subsubsection{Algorithm}

For each bubble in a time frame, the link towards the next frame is chosen as follow:
\begin{enumerate}
	\itemsep 0em
	\item Among the 3D bubbles in the next frame, find the one closest to the current position.
	\item Evaluate the distance $d$ between the current and selected position with respect to a threshold $T$:
	      \begin{itemize}
		      \item If $d>T$, it's not plausible that the bubble has moved such distance in such short time: the original bubble is likely lost, and the selected one is probably a different bubble. As such, consider the tracklet to end at the current frame;
		      \item If instead $d\le T$, the movement is plausible, therefore the two bubbles are linked.
	      \end{itemize}
\end{enumerate}

\subsubsection{Evaluation}

The simplicity of this algorithm makes it extremely fast, with the potential to reach 5000 FPS.
The main advantage is however the reconstruction quality: in the evaluation, only about 4800 tracklets were created.
 \newpage

\section{Final choice}

Differently from the other steps, all approaches of \linkDDD* had sufficient speed.
As such, the choice fell on the simple Nearest Neighbor, which had better quality.
