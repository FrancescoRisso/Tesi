\chapter{The \match* step}
\label{chap:matching}

The \match* step has the objective of leveraging the different camera views to estimate the depth (and, consequently, the 3D coordinates) of the bubbles.
The depth estimation is done with the technique of stereoscopy (explained in section~\ref{sec:backgr:stereo}).
In particular, the \match* step needs to match the same bubble across different cameras, to then call \texttt{OpenCV}'s \texttt{triangulatePoints} function for reconstructing the 3D positions.

\section{Requirements}

\subsection{Input}

Depending on the pipeline order, the input to the \match* step can be either the output of the \locate* step or the \linkDD* step.
In both cases, the input is composed of two arrays:
\begin{itemize}
	\itemsep 0em
	\item the array of the coordinates \texttt{positions[C][F][B]} (described in sections~\ref{sec:locate:output} and \ref{sec:linkDD:output}), which has the same format in both cases;
	\item the representation of valid coordinates, which varies based on the previous step: \locate* has a \texttt{numTracers[C][F]}, as explained in section~\ref{sec:locate:output}, while \linkDD* uses a larger \texttt{validTracers[C][F][B]}, described in section~\ref{sec:linkDD:output}.
\end{itemize}
In both cases, the \match* operates on the coordinates of the valid 2D coordinates: an initial step extracts such coordinates from the \texttt{positions} array, leveraging the other array in the correct way.
After that, the \match* can be executed without caring about the source of the data.

\subsection{Output}
\label{sec:match:output}

The \match* step's output consists in a couple of arrays, similar to the output of the \linkDD* (described in section~\ref{sec:linkDD:output}).

The 3D coordinates of the bubbles are stored in a three-dimensional, floating-point array called \texttt{positions}. \texttt{positions[F][B]} contains the coordinates of the $B$-th bubble in the $F$-th time instant, as a tuple \texttt{(x, y, z)}.
The camera index disappeared, since this step combines the information from all cameras into a single, 3D description of the bubbles.

To represent the valid bubbles, a three-dimensional, boolean \texttt{validTracers} array is used.
\texttt{validTracers[F][B]} marks whether \texttt{positions[F][B]} contains a valid position or not.
Similarly to the \texttt{positions} array, the passage from 2D to 3D removes the dimension of the camera index.

Depending on the source of data, this step can either produce plain 3D coordinates, or 3D tracklets.
If the input is the \locate* step, then the bubble coordinates will be all grouped towards smaller $B$s, and there will be no correlation between bubbles with the same index in consecutive frames.
Instead, if the input is the \linkDD* step, the distribution of real coordinates within the $B$ dimension of the arrays will be less regular, due to the added constraint that ``same value for $B$ implies same real bubble''.

\subsection{Speed}

To respect the real time constraint, the \match* step should operate at 30 FPS.

\subsection{Quality}

The matching algorithms start from a bubble in one ``main'' camera, and try to find the matching bubble on the other cameras.
Such attempt can have three outcomes:
\begin{itemize}
	\itemsep 0em
	\item correct match: the bubble is matched to the correct one in the other camera. This is the ideal case, since it would lead to a correct 3D reconstruction;
	\item missing match: the original bubble is not matched to another one. Since the setup is composed by more than 2 cameras, a missed match is not too terrible, since it is still possible that the matches in the other cameras are correct, to have a correct 3D reconstruction;
	\item wrong match: the bubble is matched to a wrong one. This leads to certain reconstruction errors, since the \texttt{triangulatePoints} function will have either wrong or incoherent information.
\end{itemize}
The ideal matching algorithm is therefore one that produces correct matches for all bubbles, but if it's not possible, it's better to have missing matches rather than wrong matches.

\section{State of the art}

For the \match*, no existing solution was found when looking on the Internet: the matching is usually done on full images, not on single sets of coordinates.
The only starting point available for this step was the original tool developed by the research group, the objective of the acceleration.
In particular, it allows to choose between an epiline-only approach (examined in section~\ref{sec:match:epiline}) and a brute force algorithm (evaluated in section~\ref{sec:match:bruteforce}).

When exploring the Trackpy~\cite{trackpy} documentation, the phrase ``Trackpy is a Python package for particle tracking in 2D, 3D'' may lead to think that Trackpy also performs the \match* task.
Contrarily, this means that Trackpy is able to perform tracking with 3D data in input, collected for instance using the confocal microscopy technique.
As such, it is not a useful approach for this step.

\section{Approaches}

The following chapters describe the different approaches explored for the \match* step
The evaluation was performed separately for speed and quality, with the speed measured on the same dataset as the \linkDD* step.
For the quality, it was not possible to have a ground truth containing the correct match between two cameras, neither with a real dataset nor with a synthetic one.
As such, a manual classification of correct/missed/wrong matches was done.
As visible in figure~\ref{fig:match:example}, the displacement between the view of two cameras is mostly constant: checking this consistency allows a human eye to evaluate the correctness of a match.

The qualitative evaluation was performed using two Blender-generated datasets, composed by a single frame, with respectively 30 and 100 bubbles.
An attempt was done with more bubbles, but the image was too crowded to identify correct or wrong matches.

\begin{figure}[H]
	\centerline{\includegraphics[width=0.8\textwidth]{images/match-observation.png}}
	\caption{\centering An example of \match*: the bubbles seen by the main (white) camera, matched to the corresponding bubbles seen from the other two (red, green) cameras}
	\label{fig:match:example}
\end{figure}

\newpage
\subsection{Long trajectory}
\label{sec:match:traj}

The research group that commissioned this thesis was working in parallel on a way to estimate calibration data without the need of a calibration process.
Instead, the model developed tried to perform a \match* (without knowledge of the epilines), to infer the calibration data from it.
Since this task was in common among the two projects, their solution was also considered and evaluated for the scope our purpose.

\subsubsection{Algorithm}

The solution uses a Deep Learning model, Lightglue TODO, to match long trajectories seen from different cameras.
In particular, the model performed the match on a specific frame by looking at the 200-frames-long trajectories that started in the frame itself.

\subsubsection{Evaluation}

While this approach was useful in the complex situation of autocalibration, for our task it was too computationally intensive.
Indeed, the maximum speed obtainable with this approach was 3 FPS, too far from the target 30 FPS.
 \newpage
\subsection{Closest to epiline}
\label{sec:match:epiline}

As described in section~\ref{sec:backgr:stereo:epilines}, a specific point in a camera will be seen on a specific epiline on another camera.

In traditional stereoscopy, the epiline is a set of discrete pixels, while in our case it's a continuous line of floating-point values.
On top of that, the many pixels of each bubble in the image get compressed into a single point, without size.
Due to these two factors, it is impossible to have a bubble perfectly on top of the epiline: instead, the match is selected as the closest bubble to the epiline, provided that the distance is under a reasonable threshold.

This is one of the approaches originally used in the MATLAB script provided by the research group.
It was considered as a candidate algorithm, and as such it was re-implemented in Python for easier comparison.

\subsubsection{Algorithm}

Chosen a ``main'' camera, the algorithm processes each other camera with the following steps:
\begin{enumerate}
	\itemsep 0em
	\item For each bubble in the main camera:
	      \begin{enumerate}
		      \item Compute the coefficients $a$, $b$ and $c$ of the epiline $ax+by+c{=}0$;
		      \item Compute the distance (with a scale factor) of all bubbles $(x_i, y_i)$ in the side camera from the epiline: $d_i = ax_i + by_i + c$;
		      \item Select as candidate match the bubble with $d=min_i(d_i)$;
		      \item Compare $d$ with the reasonability threshold $T$:
		            \begin{itemize}
			            \item If $d<T$, consider the bubble as a valid match;
			            \item Otherwise, consider the original bubble as unmatched.
		            \end{itemize}
	      \end{enumerate}
\end{enumerate}

\subsubsection{Evaluation}

The algorithm had excellent speed, running at 300 FPS, which is 10x faster than the requirements.
While the quality looked quite good with the smaller dataset (with 28/0/2 correct/missed/wrong matches), the performance was much worse with more bubbles (51/4/45).
As such, there was space for improvement, trading some of the useless speed with better quality.
In fact, a possible improvement could be to have some more data to describe the bubbles: the match could be chosen as the bubble with the most similar description, among those close to the epiline.
 \newpage
\subsection{Epilines + median correction}
\label{sec:match:epi-median}

As previously stated, the offset between two camera views of the same bubble is consistent across all the bubbles in the scene.
This fact is used in this approach to find and potentially correct mistakes.
Originally, the average displacement was considered as comparison term, but it would be very sensitive to errors.
Instead, the median displacement is computed, as a vector whose components are the medians of the displacement componets.
As an example, displacements of $(10,)$, $(11,0)$ and $(-3, -1)$ would generate a median displacement of $(10, 0)$.

\subsubsection{Algorithm}

For each bubble in the main camera, each other camera is processed with the following steps:
\begin{enumerate}
	\itemsep 0em
	\item For each bubble in the main frame:
	      \begin{enumerate}
		      \item Compute the coefficients $a$, $b$ and $c$ of the epiline $ax+by+c{=}0$;
		      \item Compute the distance (with a scale factor) of all bubbles $(x_i, y_i)$ in the side camera to the epiline: $d = ax_i + by_i + c$;
		      \item Select as candidate match the bubble with $d=min_i(d_i)$;
		      \item Compare $d$ with the reasonability threshold $T$:
		            \begin{itemize}
			            \item If $d<T$, consider the bubble as a valid (temporary) match;
			            \item Otherwhise, consider the original bubble as unmatched.
		            \end{itemize}
	      \end{enumerate}
	\item Compute the median of all the matches of the frame;
	\item For each bubble in the main frame:
	      \begin{enumerate}
		      \item Compute the displacement and compare it with the median:
		            \begin{itemize}
			            \item If both the difference in length and the angle between the vectors are under specific thresholds, confirm the match, and continue with the next bubble;
			            \item Otherwhise, correct the match by performing the following steps;
		            \end{itemize}
		      \item With a procedure similar to steps 1.a to 1.d, find the $N$ (parameter) bubbles closest to the epiline;
		      \item Among them, find the one whose displacement is the most similar to the median displacement;
		      \item Check the correctness of this new match:
		      \begin{itemize}
				\item If the bubble is further than the threshold $T$ (step 1.d) from the epiline, consider the match wrong and remove it;
				\item If the new displacement still does not satisfy the distance and angle from the median, consider the match wrong and remove it;
				\item Otherwhise, correct the previous match with this one.
			  \end{itemize}
	      \end{enumerate}
\end{enumerate}

\subsubsection{Evaluation}

This algorithm runs at 55 FPS, while improving on the quality: for the 30-bubbles dataset the distribution of correct/missed/wrong matches is 27/2/1, while for the 100-bubbles dataset it is 74/20/6.
 \newpage
\subsection{Epilines + short trajectory}
\label{sec:match:epi-traj}

\subsubsection{Algorithm}

\subsubsection{Evaluation}
 \newpage
\subsection{Epilines + KNN}
\label{sec:match:epi-knn}

\subsubsection{Algorithm}

\subsubsection{Evaluation}
 \newpage
\subsection{Brute force}
\label{sec:match:bruteforce}

This approach leverages the fact that more than two cameras are present, not only as a confirmation, but as a knowledge extracting method.

Given a bubble on the main camera, it should be matched to a specific bubble on the other two cameras.
If the 3D position is reconstructed from the main and one side camera, the result should be similar to the reconstuction done with the main and the other side camera.
On the other side, wrong matches would reconstruct totally different 3D coordinates.

\subsubsection{Algorithm}

\begin{enumerate}
	\itemsep 0em
	\item For each bubble in the main camera:
	      \begin{enumerate}
		      \item Reconstruct the 3D position matching it with all the bubbles on one side camera. The result will be a set of 3D points ($P_i$);
		      \item Do the same, with the other side camera (to obtain a set $P_j'$);
		      \item Find the points $p\in P_i$ and $p'\in P_j'$ that minimize their distance;
		      \item Check their distance:
		      \begin{itemize}
				\item If it is below a specific threshold, the two bubbles are considered the same, and the two matches are approved;
				\item Otherwhise, the reconstructed bubbles are the closest plausible, but still different bubbles, hence the match is considered missing.
			  \end{itemize}
	      \end{enumerate}
\end{enumerate}

\subsubsection{Evaluation}

This approach was not evaluated directly, since a very simple addition could highly reduce its computational cost.
This algorithm was therefore only considered as starting idea for the approach described in the next paragraph.
 \newpage
\subsection{Epilines + brute force}
\label{sec:match:bruteforce}

\subsubsection{Algorithm}

\subsubsection{Evaluation}
 \newpage

\section{Final choice}

Figure~\ref{fig:match:comparison} compares speed and quality of the different approaches.
Due to the real-time constraint, the long trajectories and the brute force algorithms are discarded.
Among the remaining approaches, the median approach is the best one, hence it is the selected one. 

\begin{figure}
	\centerline{\includegraphics[width=0.9\textwidth]{images/3d-matching-comparison-without-chosen.png}}
	\caption{\centering Comparing speed and quality of the various \match* approaches}
	\label{fig:match:comparison}
\end{figure}
