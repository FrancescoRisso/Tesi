\chapter{Introduction}
\label{chap:intr}

\section{Particle tracking}

Thanks to the existence of the Earth's atmosphere, we live in an enormous fish bowl, filled to the brim with air.
In our everyday life, we constantly perceive this mass of air: for instance we breathe it, we are subjected to the atmospheric pressure, we see insects flying through it, as if they were swimming.

Sometimes we can perceive that this giant mass is moving, for example when there is a slight breeze blowing on us.
This may lead curious people to desire to understand deeply how and why this movement happens.

Since air is transparent, we cannot base our research on observing it.
The tactile feelings are not enough, either, since we are only able to perceive movements of a certain strength.
We can however leverage our good vision to observe small particles floating in the air: they move with the surrounding, but unlike the air, we can see them.

Directly observing the particles can be a starting point to understand the air movement.
However, this method lacks the possibility to ``pause'' the time to reflect, and the possibility to rollback the observation to check what has just happened.
The most common solution to this problem is to record the observation with cameras, to be able to pause and re-watch the footage.
However, a simple 2D recording cannot encode the essential information about depth: for these applications, it is therefore crucial to capture the experiment in a way that allows to reconstruct the trajectories in full 3D.

While being easy for the human brain, the task of recreating a 3D scene from some 2D views is not trivial for an algorithm.
In my thesis, I had to try to accomplish it with good quality, with very stringent requirements about computation time.

\section{Structure of the thesis}

% Choose the best format

% \newcommand{\structuretype}{itemize}
% \newcommand{\structureitem}[2]{\item \textbf{Chapter #1} #2}

\newcommand{\structuretype}{description}
\newcommand{\structureitem}[2]{\item[Chapter #1] #2}

This thesis is structured as follows:

\begin{\structuretype}
	\structureitem{\ref{chap:goal}}{explains the overall goals of the project and of my thesis;}
	\structureitem{\ref{chap:background}}{provides theoretical knowledge about camera calibration and the technique of stereoscopy, and illustrates the main programming languages, libraries and tools used in the thesis work;}
	\structureitem{\ref{chap:experim-setup}}{describes the hardware where the solutions were compared and tested;}
	\structureitem{\ref{chap:basicpipeline}}{explains the steps in which the particle tracking pipeline is commonly split;}
	\structureitem{\ref{chap:locate}}{deeply analyses the requirements, the alternatives and the final implementation for the Locate step of the algorithm;}
	\structureitem{\ref{chap:2dlink}}{performs a similar analysis for the 2D link step;}
	\structureitem{\ref{chap:matching}}{examines in a similar way the approaches for 3D matching;}
	\structureitem{\ref{chap:3dlink}}{describes the 3D linking alternatives, in contrast to the 2D ones explained in chapter \ref{chap:2dlink};}
	\structureitem{\ref{chap:visual}}{illustrates the two visualizers that were developed to interactively see the reconstructed trajectories;}
	\structureitem{\ref{chap:pipeline}}{explains how the steps described in the previous chapters are joined together in a single pipeline;}
	\structureitem{\ref{chap:results}}{evaluates the quality and speed of the final solution;}
	\structureitem{\ref{chap:conclusions}}{draws the final conclusions from the thesis, and hints at possible future improvements.}
\end{\structuretype}
