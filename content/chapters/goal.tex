\chapter{Goals}
\label{chap:goal}

\section{Project overview}

Not long after the COVID-19 pandemic started, people realized that the disease could spread by means of water droplets suspended in the air.
This lead many research groups in the fluid dynamics domain to investigate how air would move inside a room, carrying such droplets (some examples:~\cite{covid-air-1}\cite{covid-air-2}).

The overall project that includes my thesis is a research topic on this momentum.
The final goal is to understand the common patterns that air follows when moving --- while apparently being still --- in a room.
To understand this, an experimental setup was created in a small room.
In a corner of the chamber there was a machine able to create bubbles, similar in concept to the soap bubbles that children use to play (figure~\ref{fig:experimetal-setup}). By observing the movement of these bubbles, the movement of the air would then be inferred.

Since the bubbles need to move in the same way of the surrounding air, a way to cancel out all the other forces is required.
The surface and filling material for the bubbles therefore need to be carefully chosen, in order to have an overall weight density of the bubble similar to the air density.
This allows the buoyancy force to compensate almost exactly the gravity force.

The experiments are conducted in two steps.
First, some bubbles are created: when there are enough, the machine is stopped, to avoid air currents caused by the machine itself.
Then, a small amount of time is waited, to allow the bubbles to lose their initial speed, and to settle in the room air movement.
After this this short period, the observation starts.
Due to this composed procedure, the ``bubble material'' would be required to create long-lasting bubbles, whose average lifetime is at least 5 minutes.

On top of that, soap bubbles are too big for the purpose: there is a high chance that a bubble in front occludes another bubble in the back, reducing the quality of the observation.
For this reason, the ``soap'' must be a material that creates bubbles with a maximum diameter of some millimeters.

From all these considerations, the bubbles were created with a coating made of [TODO CONTROLLARE IL NOME], filled with helium.

\fig{images/experimental-setup}{experimetal-setup}{The experimental setup: the machine in the corner of the room is creating some bubbles that fill the space}

\section{Project objectives}

\section{Thesis objectives}
