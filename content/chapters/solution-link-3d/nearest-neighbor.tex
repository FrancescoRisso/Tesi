\subsection{Nearest neighbor}
\label{sec:link3d:NN}

Moving from 2D space to 3D space, the bubbles are naturally much sparser.
As such, the 3D bubbles are far enough away from each other, that the nearest neighbor is undoubtedly the correct link.

\subsubsection{Algorithm}

For each bubble in a time frame, the link towards the next frame is chosen as follow:
\begin{enumerate}
	\itemsep 0em
	\item Among the 3D bubbles in the next frame, find the one closest to the current position.
	\item Evaluate the distance $d$ between the current and selected position with respect to a threshold $T$:
	      \begin{itemize}
		      \item If $d>T$, it's not plausible that the bubble has moved such distance in such short time: the original bubble is likely lost, and the selected one is probably a different bubble. As such, consider the tracklet to end at the current frame;
		      \item If instead $d\le T$, the movement is plausible, therefore the two bubbles are linked.
	      \end{itemize}
\end{enumerate}

\subsubsection{Evaluation}

The simplicity of this algorithm makes it extremely fast, with the potential to reach 5000 FPS.
The main advantage is however the reconstruction quality: in the evaluation, only about 4800 tracklets were created.
