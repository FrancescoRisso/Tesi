\section{Stereoscopy}
\label{sec:backgr:stereo}

Our brain is able to understand depth by leveraging the fact that we have two eyes in slightly different positions.
This mechanism is called stereoscopy, and it can be used by a computer vision system, provided that it has two or more cameras.

\begin{figure}
	\centerline{\includegraphics[width=0.8\textwidth]{images/epipolarity.png}}
	\caption{\centering Reprojecting a point from 2D to 3D: it could be anywhere on a specific line}
	\label{fig:epipolarity}
\end{figure}

\subsection{Depth estimation}

Consider a camera $A$, with center of projection $C_A$ and image plane $\pi_A$ (left in figure~\ref{fig:epipolarity}).
If a 3D point $P$ is seen by the camera, in the image plane it will be $P_A = \pi_A \cap \overline{PC_A}$.
From a single camera, it is impossible to reconstruct $P$ from $P_A$: there would be infinitely many possible $P$s, all the points that lie on the extension of $\overline{P_AC_A}$.

If another camera $B$ is available (right in figure~\ref{fig:epipolarity}), and the same point $P$ is projected as $P_B$, then a new information is added: that $P$ will lie on the extension of $\overline{P_BC_B}$.
By intersecting these lines, it is ideally possible to reconstruct the original 3D position of $P$.

In order to do so, the relative position of the cameras needs to be known: all calibration parameters are required.
In particular, the function \texttt{triangulatePoints} from \texttt{OpenCV} is able to reconstruct the 3D positions given the 2D matched observations, the intrinsic and distortion parameters of the two cameras, and the extrinsic matrix of the couple.

\subsection{Epilines}
\label{sec:backgr:stereo:epilines}

As stated before, the point $P_A$ could correspond to a full line of 3D points.
When seen by camera $B$ (with a different point of view), this line translates to a 2D line in $\pi_B$ (red in figure~\ref{fig:epipolarity}).
This line is called \textbf{epiline}.

Different points $P_A$ will correspond to different epilines, which however all pass by the \textbf{epipoint} $e_B$.
The epipoint is defined as $e_B = \pi_B \cap \overline{C_AC_B}$.

\subsubsection{Computing the epiline equation}

As explained in the previous section, the essential matrix $E$ is such that $P_B \cdot E \cdot P_A^T = 0$, which is called the \textbf{epipolar constraint}.
If $P_B$ is a generic point on the image, it can be described as $P_B = \rowvecthree{x}{y}{1}$.
The result of $E \cdot P_A^T$ is a $3\times 1$ vector, that can be written without loss of generality as $\rowvecthree{a}{b}{c}^T$.
When all this knowledge is substituted into the epipolar constraint, we obtain the following:
\begin{equation}
	\rowvecthree{x}{y}{1} \cdot E \cdot P_A^T = 0\\
\end{equation}
\begin{equation}
	\rowvecthree{x}{y}{1} \cdot \colvecthree{a}{b}{c} = 0\\
\end{equation}
\begin{equation}
	ax + by + c = 0
\end{equation}
which is the equation of the epiline in $B$'s frame.
% When this information is substituted into the epipolar constraint, the result of the product becomes the equation $ax + by + c = 0$, where $a$, $b$ and $c$ come from $E \cdot P_A^T$.
% Therefore, the epipolar line is 
