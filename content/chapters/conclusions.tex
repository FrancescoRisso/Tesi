\chapter{Conclusions}
\label{chap:conclusions}

As discussed in the previous chapter, our solution is a pipeline able to track 3D bubbles moving in the air, at a speed up to 38 FPS.
The output quality is excellent with up to 100 bubbles in the field of view, while it decreases with denser setups.

\section{Future work}

\subsection{Improving the speed}

While the speed is currently up to the requirements on the Jetson Orin Nano, an even faster system could run on less powerful and cheaper devices.
As discussed in section~\ref{sec:results:speed}, the main bottleneck for the speed is the loading of the images to the RAM of the processing device.
This could be improved using dedicated hardware, or by spawning different processes to load the different images concurrently.
For this last proposal, a study on the specific final device should be done, to be able to choose the right amount of processes.
It would be crucial to balance the added parallelism with the cost of context switching, in case there are not enough cores to execute all the processes concurrently.

\subsection{Improving the quality}

Depending on the nature of the experiment, more than 100 bubbles may be required for the complete understanding of the phenomenon under observation.
In order to increase the amount of bubbles reconstructed with good quality, the \match* step should be improved.

Since the ``Epilines + brute force'' approach is the one with the best results, and it is not far from the target speed, some more experiments could be performed to try to accelerate it to an acceptable speed.
This would improve the result, by having a better matching.

Another option would be to add more information to the data given to the \match* algorithm.
An example would be if the bubbles could have different colors: if, for example, half the real bubbles were yellow and half were blue, the matching would only need to choose among half of the current candidates, the ones with the same color.
In general, adding an extra piece of information that splits the real bubbles into $N$ sets would on average enable to multiply by $N$ the number of bubbles that can be reconstructed with good quality.
Another alternative to the color could be the thermal information, if the temperature of the bubble changes during the time it spends in the air.
For both these ideas, however, a different hardware setup should be constructed: RGB or thermal cameras would be required, and the images could not be simply binary, but would need more information.
