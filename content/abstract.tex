% abstract, choose between abstract and summary

For better understanding what happens during an experiment, and extracting as much information as possible from it, the possibility of recording the experiment itself and watching it again at a later time is crucial.
% The possibility of recording precisely the behaviour of a given experiment and watching it again at a later time is crucial for better understanding what happened and extracting as much information as possible.
%The possibility of watching again what happened is crucial for extracting as much information as possible from an experiment.
While for many cases achieving this is just a matter of recording the event with a camera, some complex situations require more attention.
One such case is when the nature of the experiments is to understand 3D data, where a 2D video would not carry the necessary information.

% In this thesis, we develop a full pipeline to process a set of synchronized videos, to reconstruct the 3D trajectory of small bubbles moving in the air.
This thesis aims at recording 3D trajectories of small bubbles moving in the air.
%In this thesis, we want to record 3D trajectories of small bubbles moving in the air.
This is achieved by combining the information provided by a set of synchronized cameras, observing the experiment.
The source videos capture the same scene from different positions, which are known thanks to an initial calibration process.
The special-purpose cameras are able to provide binary frames, highlighting the bubbles in white, on top of a black background.
The proposed solution starts by transforming each frame into a list of 2D coordinates, describing where the center of each bubble is within the image.
For each time instant, the different points of view are then leveraged to reproject the coordinates into the 3D space, creating a cloud of 3D points.
Subsequently, the intrinsic sequential nature of the video is made explicit: reconstructed bubbles of consecutive frames are joined together, to form trajectories.
Finally, the obtained 3D reconstruction needs to be displayed on a 2D monitor: different techniques are exploited to make this visualization as intuitive and understandable as possible.

This thesis illustrates all the approaches evaluated for each step, and which ones are eventually chosen thanks to their speed and quality performance.
The final solution is a pipeline that can process in real-time the experimental setup, displaying the reconstructed trajectories only a few seconds after the image capture.
